\documentclass{exercises}

\DeclareMathOperator{\ev}{ev}
\DeclareMathOperator{\Fix}{Fix}
\DeclareMathOperator{\Idl}{Idl}
\DeclareMathOperator{\im}{im}
\DeclareMathOperator{\Mat}{Mat}
\DeclareMathOperator{\Nil}{Nil}
\DeclareMathOperator{\Rad}{Rad}
\DeclareMathOperator{\Specm}{Spec_m}
\DeclareMathOperator{\Spec}{Spec}
\DeclareMathOperator{\tr}{tr}

\begin{document}
\maketitle{Universität Augsburg}{Lehrstuhl für Algebra und Zahlentheorie}{Wintersemester 2021/22}{Dr.~Marco Ramponi}{Lukas Stoll, M.Sc.}{Algebraic Geometry I}

\begin{exercise}[Galois connections]
  A \emph{partially ordered set}, or short \emph{poset}, is a set $A$ equipped with a reflexive, transitive and antisymmetric relation $≤$.
  A map $f:A → B$ between partially ordered sets is \emph{order preserving} if $a ≤_A a'$ implies $f(a) ≤_B f(a')$.

  A \emph{Galois connection} between two partially ordered sets is a pair of order preserving maps $g:(B,≤_B) ⇆ (A,≤_A):f$ which satisfies the following \emph{adjunction property} for all $a ∈ A$, $b ∈ B$:
  $$
  f(a) ≤_B b \quad\Longleftrightarrow\quad a ≤_A g(b)
  $$
  \begin{enumerate}
    \item Let $X$, $Y$ be sets and $R ⊆ X × Y$ a relation from $X$ to $Y$.
      Prove that the following maps form a Galois connection between the power sets of $X$ and $Y$.
      \begin{align*}
        𝒱_R : (𝒫(X),⊇) &\leftrightarrows (𝒫(Y),⊆) : ℐ_R\\
        S & \longmapsto \{y ∈ Y \mid ∀s ∈ S:s \perp y\}\\
        \{x ∈ X \mid ∀t ∈ T:x \perp t\} & \longmapsfrom T
      \end{align*}
      {\scriptsize Note that $≤_A$ is the subset relation $⊆$ and $≤_B$ the reverse subset relation $⊇$, making order preserving maps between the two powersets \emph{inclusion reversing}.}
    \item Let $K$ be a field.
      Show the existence of a Galois connection
      $$
      𝒱:(𝒫(K[x_1,\dots,x_n]),⊇) ⇆ (𝒫(𝔸^n_K),⊆):ℐ
      $$
      between the power sets of the polynomial ring $K[x_1,\dots,x_n]$ and the affine space $𝔸^n_K$, such that $𝒱(S)$ is the vanishing set of $S⊆K[x_1,\dots,x_n]$ and $ℐ(T)$ the vanishing ideal of $T⊆𝔸^n_K$.
    \item Let $\Idl(R)$ denote the set of ideals of a commutative ring $R$.
      Show that $ℐ(T)$ is an ideal for every $T⊆𝔸^n_K$ and conclude that the above Galois connection descends to a Galois connection $𝒱:(\Idl(K[x_1,\dots,x_n]),⊇) ⇆ (𝒫(𝔸^n_K),⊆):ℐ$.
    \item Let $\Rad(R)$ denote the set of radical ideals of a commutative ring $R$.
      Show that $ℐ(T)$ is radical for every $T⊆𝔸^n_K$ and conclude that the above Galois connection descends to a Galois connection $𝒱:(\Rad(K[x_1,\dots,x_n]),⊇) ⇆ (𝒫(𝔸^n_K),⊆):ℐ$.\\
      {\scriptsize An ideal $𝖆 ⊆ R$ is \emph{radical} if $∀f ∈ R : (∃n ∈ ℕ : f^n ∈ 𝖆) ⇒ f ∈ 𝖆$.}
  \end{enumerate}
\end{exercise}

\begin{exercise}[Closure operators]
  A \emph{closure operator} on a poset $A$ is an order preserving map $◯:A → A$ which is \emph{idempotent}, $◯(◯(a)) = ◯(a)$, and \emph{extensive}, $a ≤ ◯(a)$.

  Let $g:B ⇆ A:f$ be a Galois connection.
  \begin{enumerate}
    \item Show that $g∘f$ is a closure operator on $(A,≤_A)$.
    \item Show that $f∘g$ is a closure operator on $(B,≥_B)$.
      {\scriptsize Mind the reverse ordering!}
  \end{enumerate}
  Let $R$ be a commutative ring.
  Prove that the following maps are closure operators.
  \begin{enumerate}[start=3]
    \item The map $(\_):𝒫(R) → 𝒫(R)$ sending a subset $S⊆R$ to the ideal it generates. 
      $$
      (S)\coloneqq \{r_1s_1 + \dots + r_ns_n \mid n ∈ ℕ,\, r_i ∈ R,\, s_i ∈ S\}
      $$
    \item The map $\sqrt{-}:\Idl(R) → \Idl(R)$ sending an ideal $𝖆⊆R$ to its radical.
      $$
      \sqrt{𝖆}\coloneqq\{r ∈ R \mid ∃n ∈ ℕ : r^n ∈ 𝖆\}
      $$
  \end{enumerate}
  Let $K$ be a field.
  Prove:
  \begin{enumerate}[start=5]
    \item $𝒱((S))=𝒱(S)$ for every subset $S⊆K[x_1,\dots,x_n]$.
    \item $𝒱(\sqrt{I})=𝒱(I)$ for every ideal $I⊆K[x_1,\dots,x_n]$.
  \end{enumerate}
\end{exercise}

\begin{exercise}[Galois correspondences]
  A Galois connection $g:B ⇆ A:f$ is a \emph{Galois correspondence} if $f$ and $g$ are mutual inverses.
  \begin{enumerate}
    \item Show that a Galois connection induces a Galois correspondence between the the sets of fixed points of its associated closure operators.
      \begin{equation*}
        \begin{tikzcd}
          (\Fix(f∘g),≤_B)
          \arrow[r,yshift=-2.5,"g"']
          & (\Fix(g∘f),≤_A)
          \arrow[l,yshift=2.5,"f"', "\sim" yshift=0.5pt]
        \end{tikzcd}
      \end{equation*}
      {\scriptsize A fixed point of an endomap $s:X → X$ is an element $x ∈ X$ with $s(x)=x$.}
  \end{enumerate}
  Let $K$ be an algebraically closed field.
  \begin{enumerate}[start=2]
    \item Find a Galois correspondence between the radical ideals of $K[x_1,\dots,x_n]$ and the affine varieties in $𝔸^n_K$.
      {\scriptsize Use Hilbert's Nullstellensatz.}
    \item Show that this Galois correspondence restricts to a Galois correspondence between the maximal ideals of $K[x_1,\dots,x_n]$ and the points in $𝔸^n_K$.
  \end{enumerate}
\end{exercise}

\begin{exercise}[Joins and meets]
  Let $(A,≤)$ be a poset and $M⊆A$.
  An upper bound $a ∈ A$ of $M$ (that is, $m ≤ a$ for all $m ∈ M$) is called a \emph{join} or \emph{supremum} of $M$ if it satisfies the following universal property:
  $$
  ∀a' ∈ A: (∀m ∈ M:m ≤ a') ⇒ a ≤ a'
  $$
  Dually $a$ is a \emph{meet} or \emph{infimum} of $M$ if it is a join of $M$ in the poset $(A,≥)$ with reverse ordering.
  If each finite subset has a join and a meet, $A$ is called a \emph{lattice}.
  If this holds for all subsets, the lattice is said to be \emph{complete}.

  \begin{enumerate}
    \item Show that meets and joins are unique if they exist.  In case of existence, write $⋁ M$ for the join of $M$ and $⋀ M$ for its meet.
    \item Show that the following are equivalent for a poset $A$:
      \begin{enumerate}
        \item $A$ is complete.
        \item Every subset of $A$ has a meet.
        \item Every subset of $A$ has a join.
      \end{enumerate}
    \item Let $◯:A → A$ be a closure operator on a poset $(A,≤)$.
    Let $M⊆\Fix(◯)$ and regard $\Fix(◯)$ as a subposet of $(A,≤)$.
      Prove:
      \begin{enumerate}
        \item A meet of $M$ in $A$ is a fixed point of $◯$.
          Moreover, it is the meet of $M$ in $\Fix(◯)$.
        \item If $M$ has a join in $A$, its image under $◯$ is the join of $M$ in $\Fix(◯)$.
        \item Deduce that $\Fix(◯)$ is a complete lattice if $A$ is.
        \item Prove that $◯(a) = ⋀ \{m ∈ \Fix(◯) \mid a ≤ m\}$ for all $a ∈ A$.
      \end{enumerate}
    \item Compute meet and join of
      \begin{enumerate}
        \item a set $M⊆𝒫(X)$ of subsets of a set $X$,
        \item a set $M⊆\Idl(R)$ of ideals of a commutative ring $R$,
        \item a set $M⊆\Rad(R)$ of radical ideals of a commutative ring $R$,
      \end{enumerate}
      and conclude that $𝒫(X)$, $\Idl(X)$ and $\Rad(X)$ are complete lattices.
  \end{enumerate}
\end{exercise}

\begin{exercise}[Preservation of joins and meets]
  Let $g:B ⇆ A:f$ be a Galois connection.
  Show:
  \begin{enumerate}
    \item The right adjoint $g$ preserves meets: $g \left( ⋀ M \right) = ⋀ g(M)$ for all $M⊆B$.
    \item The left adjoint $f$ preserves joins: $f \left( ⋁ N \right) = ⋁ f(N)$ for all $N⊆A$.
  \end{enumerate}
  Let $K$ be a field, $(𝖆_i)_{i ∈ I}$ a family of ideals of $K[x_1,\dots,x_n]$, and $(T_i)_{i ∈ I}$ a family of subsets of $𝔸^n_K$.
  Prove the following equalities:
  \begin{enumerate}[start=3]
    \item $𝒱(∑𝖆_i) = 𝒱(⋃𝖆_i) = ⋂𝒱(𝖆_i)$.
    \item $ℐ(⋃ T_i) = ⋂ℐ(T_i)$.
  \end{enumerate}
  Finally, suppose $𝖆,𝖇 ⊆ K[x_1,\dots,x_n]$ are ideals.
  \begin{enumerate}[start=5]
    \item Prove that $𝒱(𝖆 ∩ 𝖇) = 𝒱(𝖆 ⋅ 𝖇) = 𝒱(𝖆) ∪ 𝒱(𝖇)$.
  \end{enumerate}
\end{exercise}

\begin{exercise}[Ideal operations]
  Let $R$ be a commutative ring and $𝖆,𝖇,𝖈⊆R$ ideals.
  Recall the definition of the \emph{ideal sum}
  $$
  𝖆 + 𝖇 \coloneqq (𝖆 ∪ 𝖇),
  $$
  the \emph{ideal product}
  $$
  𝖆 ⋅ 𝖇 \coloneqq (\{a ⋅ b ∈ R\mid a ∈ 𝖆, b ∈ 𝖇\}),
  $$
  and the \emph{ideal quotient}
  $$
  𝖆 : 𝖇 \coloneqq \{r ∈ R \mid r ⋅ 𝖇 ⊆ 𝖆\}.
  $$
  Show:
  \begin{enumerate}
    \item The intersection $𝖆 ∩ 𝖇$ is an ideal but the union $𝖆 ∪ 𝖇$ is not in general.
    \item The quotient $𝖆 : 𝖇$ is an ideal but $\{a ⋅ b ∈ R \mid a ∈ 𝖆, b ∈ 𝖇\}$ is not in general.
    \item There is a chain of inclusions $𝖆 ⋅ 𝖇 ⊆ 𝖆 ∩ 𝖇 ⊆ 𝖆 ⊆ 𝖆 + 𝖇$.
      Find counterexamples to show that the inclusions are strict in general.
    \item $\sqrt{𝖆 ⋅ 𝖇} = \sqrt{𝖆 ∩ 𝖇} = \sqrt{𝖆} ∩ \sqrt{𝖇}$.
    \item $\sqrt{𝖆 + 𝖇} = \sqrt{\sqrt{𝖆} + \sqrt{𝖇}}$.
  \end{enumerate}
\end{exercise}

\begin{exercise}
  Let $R$ be a ring and $𝖆,𝖇⊆R$ ideals.
  Two ideals $𝖆,𝖇⊆R$ are \emph{comaximal} if $𝖆 + 𝖇 = R$.
  \begin{enumerate}
    \item Let $𝖆,𝖇$ be comaximal ideals.
      Prove:
      \begin{enumerate}
        \item $𝖆⋅𝖇 = 𝖆 ∩ 𝖇$.
        \item $R/(𝖆⋅𝖇) ≅ (R/𝖆) × (R/𝖇)$.
      \end{enumerate}
    \item Let $𝖆$ be comaximal to $𝖇$ and $𝖇'$.
      Prove that $𝖆$ is comaximal to $𝖇⋅𝖇'$.
    \item Let $𝖆$, $𝖇$ be comaximal and $m,n≥1$.
      Prove that $𝖆^m$ and $𝖇^n$ are comaximal.
    \item Let $𝖆_0,\dots,𝖆_n$ be pairwise comaximal.
      Prove:
      \begin{enumerate}
        \item $𝖆_0$ and $𝖆_1 \cdots 𝖆_n$ are comaximal.
        \item $𝖆_0∩\dots∩𝖆_n = 𝖆_0 \cdots 𝖆_n$.
        \item $R/(𝖆_0\cdots 𝖆_n) ≅ (R/𝖆_0) × \cdots × (R/𝖆_n)$.
      \end{enumerate}
  \end{enumerate}
\end{exercise}

\begin{exercise}[Counting monomials]
  Let $K$ be a field.
  Prove that the number of monomials of degree $d$ in $n$ variables over $K$ is given by the binomial coefficient $\binom{d+n-1}{n-1}$.\\
  {\scriptsize This is the dimension of the vector space $K[x_1,\dots,x_n]_d$ of polynomials over $K$ in $n$ variables of degree $d$.}
\end{exercise}

\begin{exercise}[Fun with monomial orders]
  List all $20$ monomials in three variables $K[x,y,z]$ of degree $≤3$ 
  \begin{enumerate}
    \item in the lexicographical order,
    \item in the deg-lex order,
    \item in the deg-rev-lex order.
  \end{enumerate}
\end{exercise}

\begin{exercise}[The Zariski topology]
  Let $(A,≤)$ be a partially ordered set.
  A closure operator $◯:A → A$ is called \emph{topological} or \emph{Kuratowski} if it preserves finite joins.
  \begin{enumerate}
    \item Suppose $A$ is the power set of a set $X$.
      Show that any topological closure operator $◯$ on $A$ induces a topology on $X$ whose closed sets are the fixed points of $◯$.
      Conversely, show that any topology on $X$ induces a topological closure operator on its power set.
    \item Let $K$ be a field.
      Show that the closure operator $𝒱∘ℐ$ on the power set of $𝔸^n_K$ is topological and that its closed sets are the affine varieties.
    \item More generally, let $R$ be a commutative ring.
      Find a Galois connection between the power set of $R$ and the power set of its spectrum such that the associated closure operator is topological and induces the Zariski topology on $\Spec R$.
    \item In the case that $R=K[x_1,\dots,x_n]$ for a field $K$, find a map $𝔸^n_K → \Spec R$ which is a subspace inclusion with respect to the Zariski topologies.
  \end{enumerate}
\end{exercise}

\begin{exercise}[Ring maps induce continuous functions]
  Let $f:R → S$ be a ring homomorphism.
  Given a prime ideal $𝖕⊆S$, let $f^*𝖕 \coloneqq f^{-1}𝖕$.
  \begin{enumerate}
    \item Show that this defines a continuous map $f^*:\Spec S → \Spec R$.
    \item Prove that $(g∘f)^*=f^*∘g^*$ for two composable ring maps $f$ and $g$.
  \end{enumerate}
\end{exercise}

\begin{exercise}[Polynomials and polynomial functions]
  Let $W⊆𝔸^n_K$ be an affine variety over a field $K$ and denote the $K$-algebra of maps $W → K$ by $K^W$.
  There is  a unique morphism of $K$-algebras $\ev_W:K[X_1,\dots,X_n] → K^W$ sending $X_i$ to the restriction of the $i$-th projection $𝔸^n_K → K$ to $W$.
  An element in the image of this morphism is called a \emph{polynomial function}.
  \begin{enumerate}
    \item Confirm that the kernel of $\ev_W$ is given by the vanishing ideal $ℐ(W)$ of $W$.
    \item Deduce that $\ev_W$ is injective only if $W=𝔸^n_K$.
    \item Show that $\ev_{𝔸^n_K}$ is injective only if $K$ is infinite.
    \item Does the converse hold?
  \end{enumerate}
\end{exercise}

\begin{exercise}[Connected and irreducible spaces]
  Let $X$ be a topological space.
  \begin{enumerate}
    \item $X$ is \emph{connected} if for any finite disjoint decomposition $⨆_{i=1}^nZ_i$ into closed subsets $Z_i⊆X$ there is an $i$ with $X=Z_i$.
      Prove:
      \begin{enumerate}
        \item The empty space $∅$ is not connected.
        \item The one-point space $𝟙$ is connected.
        \item The image of a connected space under a continuous function is connected.
        \item The closure of a connected subspace is connected.
        \item The union of connected subspaces of $X$ is connected if they have pairwise nonempty intersection.
      \end{enumerate}
      A subspace $Y⊆X$ which is maximal with respect to inclusion among all connected subspaces of $X$ is called a \emph{connected component} of $X$.
      Prove:
      \begin{enumerate}[start=6]
        \item Every connected subspace of $X$ is contained in a unique connected component.
        \item $X$ is the disjoint union of its connected components.
      \end{enumerate}
    \item $X$ is \emph{irreducible} if for any finite decomposition $⋃_{i=1}^nZ_i$ into closed subsets $Z_i⊆X$ there is an $i$ with $X=Z_i$.
      Prove:
      \begin{enumerate}
        \item The empty space $∅$ is not irreducible.
        \item The one-point space $𝟙$ is irreducible.
        \item The image of an irreducible space under a continuous function is irreducible.
        \item The closure of an irreducible subspace is irreducible.
        \item The union of a nonempty chain of irreducible subspaces of $X$ is irreducible.
      \end{enumerate}
      A subspace $Y⊆X$ which is maximal with respect to inclusion among all irreducible subspaces of $X$ is called an \emph{irreducible component} of $X$.
      Prove:
      \begin{enumerate}[start=6]
        \item Every irreducible subspace of $X$ is contained in an irreducible component. {\tiny Zorn's lemma.}
        \item $X$ is the union of its irreducible components.
      \end{enumerate}
    \item Show that an irreducible space is connected.
    \item Show that $ℝ^n$ is connected but not irreducible in the Euclidean topology for $n≥1$.
    \item Show that $𝔸_K^n$ is irreducible in the Zariski topology if and only if the field $K$ is infinite.
    \item Show that $\Spec R$ is irreducible if $R$ is an integral domain. {\tiny Use the next exercise.}
  \end{enumerate}
\end{exercise}

\begin{exercise}[The fundamental Galois correspondences of modern algebraic geometry]
  Let $R$ be a commutative ring.
  Consider the relation $\{(f,𝖕)\mid [f] = 0 ∈ R/𝖕\}⊆R×\Spec R$ and its induced Galois connection
  \begin{equation*}
    \begin{tikzcd}
      (𝒫(R),⊇)
      \arrow[r,yshift=-2.5,"𝒱"']
      & (𝒫(\Spec(R)),⊆)
      \arrow[l,yshift=2.5,"ℐ"']
    \end{tikzcd}
  \end{equation*}
  which induces the Zariski topology on $\Spec(R)$ (cf.~Exercise 10).
  Show that this Galois connection descends to the following Galois correspondences:
  \begin{enumerate}
    \item $
      \begin{tikzcd}[cramped]
        (\Rad(R),⊇)
        \arrow[r,yshift=-2.5]
        & (\{\text{closed subsets of }\Spec(R)\},⊆)
        \arrow[l,yshift=2.5,"\sim" yshift=0.5pt]
      \end{tikzcd}
      $
    \item $
      \begin{tikzcd}[cramped]
        (\Spec(R),⊇)
        \arrow[r,yshift=-2.5]
        & (\{\text{irreducible closed subsets of }\Spec(R)\},⊆)
        \arrow[l,yshift=2.5,"\sim" yshift=0.5pt]
      \end{tikzcd}
      $
    \item $
      \begin{tikzcd}[cramped]
        (\Specm(R),⊇)
        \arrow[r,yshift=-2.5]
        & (\{\text{closed points of }\Spec(R)\},⊆)
        \arrow[l,yshift=2.5,"\sim" yshift=0.5pt]
      \end{tikzcd}
      $
  \end{enumerate}
  Conclude, that each irreducible closed subspace $Y$ of a spectrum has a \emph{generic point}, that is, there is a $y∈Y$ such that $\overline{\{y\}} = Y$.
\end{exercise}

\begin{exercise}[Spectra see more than classical varieties]
  Let $K$ be a field and consider the polynomial $f=X^4-X^2-2 ∈ K[X]$.
  List the points of $𝒱(f)⊆𝔸^1_K$ in the affine line …
  \begin{enumerate}
    \item … over the complex numbers, $K=ℂ$,
    \item … over the real numbers, $K=ℝ$,
    \item … over the rational numbers, $K=ℚ$.
  \end{enumerate}
  Now repeat the exercise but consider $𝒱(f)$ as a subset of the spectrum $\Spec K[X]$ instead.
\end{exercise}

\begin{exercise}[Open and closed subvarieties]
  \begin{enumerate}
    \item Let $φ:A → B$ be a map of sets.
      Show that $φ^* : (𝒫(B),⊆) ⇆ (𝒫(A),⊆) : ∃_φ$ forms a Galois connection, where $∃_φ(X)\coloneqq φ(X)$ is the image of $X⊆A$ under $φ$ and $φ^*(Y)\coloneqq φ^{-1}(Y)$ is the inverse image of $Y⊆B$ under $φ$.
    \item Suppose $φ:A → B$ is a morphism of rings.
      Show that $φ^* : (\Idl(B),⊆) ⇆ (\Idl(A),⊆) : φ_!$ forms a Galois connection, where $φ_!(X)\coloneqq (∃_φ(X))$ is the ideal generated by the image of $X⊆A$ under $φ$.
      Moreover, show that $φ_!=∃_φ$ if $φ$ is surjective.
    \item Suppose $φ:A → A/𝖆$ is the projection onto the quotient of $A$ by an ideal $𝖆⊆A$.
      Show that the above Galois connection between sets of ideals restricts to a Galois correspondence
      $$
      φ^* : \Spec(A/𝖆) ⇆ 𝒱(a) : φ_!,
      $$
      Conclude that this correspondence induces a closed immersion%
      \footnote{A continuous injective map is an \emph{immersion} if it is a homeomorphism onto its image.}
      $\Spec(A/𝖆) ↪ \Spec(A)$.
    \item Suppose $φ:A → S^{-1}A$ is the localization of $A$ at a multiplicative submonoid $S⊆A$.
      Show that the above Galois connection between sets of ideals restricts to a Galois correspondence
      $$
      φ^* : \Spec(S^{-1}A) ⇆ \{𝖕 ∈ \Spec A \mid 𝖕 ∩ S = ∅\} : φ_!,
      $$
      Conclude that this correspondence induces an immersion $\Spec(S^{-1}A) ↪ \Spec(A)$.
    \item Show that the immersion $\Spec(S^{-1}A) ↪ \Spec(A)$ is open if $S$ is of the form $\{1,f,f^2,\dots\}$ for some $f ∈ A$.
      In this case we write $A_f$ for $S^{-1}A$ and $D(f)$ for the image of $\Spec(A_f)$ in $\Spec A$ under the immersion.
    \item Find an example where the immersion $\Spec(S^{-1}A) ↪ \Spec(A)$ is neither open nor closed.
    \item Now suppose that $A$ is a polynomial ring $K[X_1,\dots,X_n]$ over a field $K$ and let $f ∈ A$.
      Find a variety $V$ and a map $V → 𝔸^n_K$ which is polynomial in each entry and induces a bijection onto $D(f)∩𝔸^n_K$.
  \end{enumerate}
\end{exercise}

\begin{exercise}[Monomial ideals] 
  Let $K$ be a field.
  Recall that a \emph{monomial} is a product of variables and an ideal $𝖆 ⊆ K[X_1,\dots,X_n]$ is a \emph{monomial ideal} if it admits a generating set consisting of monomials.
  \begin{enumerate}
    \item Show that $𝖆$ is monomial if and only if for each linear combination $c_1f_1 + \dots + c_rf_r ∈ 𝖆$ of monomials $f_1,\dots,f_s$ with $c_j ∈ K^×$, we have $f_j ∈ 𝖆$ for all $j$.
    \item Show that the radical of a monomial ideal is monomial.
    \item Given a monomial $m=X_{i_1}^{e_1}\cdots X_{i_m}^{e_m}$ with $e_j > 0$, define $\sqrt{m} \coloneqq X_{i_1}\cdots X_{i_m}$.
      We call $m$ \emph{square-free} if $\sqrt{m}=m$.
      Show that a monomial ideal admits a generated set consisting of square-free monomials if and only if it is radical.
    \item Given monomials $m_1,\dots,m_r$, show that $\sqrt{(m_1,\dots,m_r)} = (\sqrt{m_1},\dots,\sqrt{m_r})$ and conclude that the vanishing set of a monomial ideal is a union of linear subspaces.
    \item Suppose $𝖆$ is a monomial ideal and denote by $S$ the smallest set of variables such that $(S)⊇𝖆$.
      Show that $\dim 𝖆 = n - |S|$.
    \item Now let $V⊆𝔸^n_K$ be a variety such that $\mathrm{in}_\prec(ℐ(V))=𝖆$.
      Can you find a geometric interpetation of the dimension formula above?
  \end{enumerate}
\end{exercise}

\begin{exercise}[Homogeneous ideals]
  Let $K$ be a field.
  Recall that a \emph{homogeneous polynomial of degree $d$} is a linear combination of monomials of degree $d$ over $K$ and an ideal $𝖆⊆K[X_0,\dots,X_n]$ is a \emph{homogeneous ideal} if it admits a generating set consisting of homogeneous elements (not necessarily of the same degree).
  Write $K[X_0,\dots,X_n]_d$ for the linear subspace consisting of homogeneous polynomials of degree $d$.
  \begin{enumerate}
    \item Assume that $K$ is infinite.
      Show that $f ∈ K[X_0,\dots,X_n]$ is homogeneous of degree $d$ if and only if $f(λa_0,\dots,λa_n) = λ^d f(a_0,\dots,a_n)$ for all $a_0,\dots,a_n ∈ K$ and $λ ∈ K^×$.
    \item Let $𝖆⊆K[X_0,\dots,X_n]$ be an ideal.
      Show that the following assertions are equivalent.
      \begin{enumerate}
        \item The ideal $𝖆$ is homogeneous.
        \item For every $f∈𝖆$ its homogeneous components are again in $𝖆$.
        \item The ideal $𝖆$ decomposes as the direct sum $𝖆 = \bigoplus_{d≥0}𝖆 ∩ K[X_0,\dots,X_n]_d$.
      \end{enumerate}
    \item Show that intersections, sums, products and radicals of homogeneous ideals are again homogeneous.
    \item Show that a homogeneous ideal $𝖕⊆K[X_0,\dots,X_n]$ is prime if and only if for any finite family $(f_i)_{i∈\{1,\dots,r\}}$ of homogeneous elements $f_i$ with $∏_{i=1}^r f_i ∈ 𝖕$ there exists an $i ∈ \{1,\dots,r\}$ such that $f_i ∈ 𝖕$.
    \item Show that every homogeneous ideal $𝖆 ⊊ K[X_0,\dots,X_n]$ is contained in the \emph{irrelevant ideal}, the homogeneous ideal $(X_0,\dots,X_n)$.
  \end{enumerate}
\end{exercise}

\begin{exercise}[Projective space as a quotient]
  \begin{enumerate}
    \item Recall the projection map $π:𝔸^{n+1}_K∖\{0\} → ℙ^n_K$, sending a point $p=(a_0,\dots,a_n)$ to the line $[a_0:\ldots:a_n]$ through $p$ and the origin.
      Show that $π$ is continuous.
    \item Consider the action of $K^×$ on $𝔸^{n+1}_K$ by scaling, that is, $λ∈K^×$ sends $(a_0,\dots,a_n)$ to $(λa_0,\dots,λa_n)$.
      Denote the corresponding quotient projection $𝔸^{n+1}_K∖\{0\} → (𝔸^{n+1}_K∖\{0\})/K^×$ by $ν$ and equip $(𝔸^{n+1}_K∖\{0\})$ with the quotient topology.
      Find a homeomorphism making the following triangle commute:
      \[\begin{tikzcd}[column sep=5em]
        & {𝔸^{n+1}_K∖\{0\}} \\
        {ℙ^n_K} && {(𝔸^{n+1}_K∖\{0\})/K^×}
        \arrow["{≃}", dashed, from=2-1, to=2-3]
        \arrow["ν", from=1-2, to=2-3]
        \arrow["{π}"', from=1-2, to=2-1]
      \end{tikzcd}\]
  \end{enumerate}
\end{exercise}

\begin{exercise}[Affine cones]
  Let $K$ be an infinite field and write $R\coloneqq K[X_0,\dots,X_n]$.
  Recall that $K$ acts on affine space $𝔸^{n+1}_K$ by scaling, that is, $λ ∈ K$ sends $(a_0,\dots,a_n)$ to $(λa_0,\dots,λa_n)$.
  A subset $C⊆𝔸^{n+1}_K$ is an \emph{affine cone} if $λ⋅p ∈ C$ for each $λ ∈ K$ whenever $p ∈ C$.
  \begin{enumerate}
    \item Denote the set of homogeneous ideals of $R$ which are not equal to the irrelevant ideal by $\Idl_+(R)$ and write $π:𝔸^{n+1}_K∖\{0\} → ℙ^n_K$ for the canonical projection.
      Show that there is a commutative triangle of Galois connections as follows:
      % https://q.uiver.app/?q=WzAsMyxbMCwxLCIoXFxJZGxfKyhSKSziiocpIl0sWzIsMCwiKFxce1xcdGV4dHtzdWJzZXRzIG9mIH3ihJlebl9LXFx9LOKKhikiXSxbMiwyLCIoXFx7XFx0ZXh0e25vbnRyaXZpYWwgYWZmaW5lIGNvbmVzIGluIH3wnZS4XntuKzF9X0tcXH0s4oqGKSJdLFswLDIsIvCdkrEiLDIseyJvZmZzZXQiOjJ9XSxbMSwyLCLPgF4qIiwyLHsib2Zmc2V0IjoyfV0sWzAsMSwi8J2SsV8rIiwyLHsib2Zmc2V0IjoyfV0sWzEsMCwi4oSQXysiLDIseyJvZmZzZXQiOjJ9XSxbMiwxLCLPgF8hIiwyLHsib2Zmc2V0IjoyfV0sWzIsMCwi4oSQIiwyLHsib2Zmc2V0IjoyfV0sWzQsNywiIiwwLHsibGV2ZWwiOjEsInN0eWxlIjp7Im5hbWUiOiJhZGp1bmN0aW9uIn19XSxbNiw1LCIiLDEseyJsZXZlbCI6MSwic3R5bGUiOnsibmFtZSI6ImFkanVuY3Rpb24ifX1dLFs4LDMsIiIsMCx7ImxldmVsIjoxLCJzdHlsZSI6eyJuYW1lIjoiYWRqdW5jdGlvbiJ9fV1d
      \[\begin{tikzcd}
        && {(\{\text{subsets of }ℙ^n_K\},⊆)} \\
        {(\Idl_+(R),⊇)} \\
        && {(\{\text{nontrivial affine cones in }𝔸^{n+1}_K\},⊆)}
        \arrow[""{name=0, anchor=center, inner sep=0}, "{𝒱}"', shift right=2, from=2-1, to=3-3]
        \arrow[""{name=1, anchor=center, inner sep=0}, "{C}"', shift right=2, from=1-3, to=3-3]
        \arrow[""{name=2, anchor=center, inner sep=0}, shift right=2, from=2-1, to=1-3]
        \arrow[""{name=3, anchor=center, inner sep=0}, shift right=2, from=1-3, to=2-1]
        \arrow[""{name=4, anchor=center, inner sep=0}, "{\widetilde{∃}_π}"', shift right=2, from=3-3, to=1-3]
        \arrow[""{name=5, anchor=center, inner sep=0}, "{ℐ}"', shift right=2, from=3-3, to=2-1]
        \arrow["\dashv"{anchor=center}, draw=none, from=1, to=4]
        \arrow["\dashv"{anchor=center, rotate=-77}, draw=none, from=3, to=2]
        \arrow["\dashv"{anchor=center, rotate=-103}, draw=none, from=5, to=0]
      \end{tikzcd}\]
      Here, $C(T)$ is the affine cone $π^{-1}(T)∪\{0\}$ associated to a subset $T⊆ℙ^n_K$ of projective space, $\widetilde{∃}_π$ sends a subset $X ⊆ 𝔸^{n+1}_K$ to $π(X∖\{0\})$.
    \item Show that the above triangle of Galois connections descends to a triangle of Galois correspondences between the set of homogeneous \textbf{radical} ideals which are not equal to the irrelevant ideal, the set of \textbf{closed} sets of $ℙ^n_K$, and the set of nontrivial \textbf{closed} affine cones in $𝔸^{n+1}_K$.
    \item Show that the above triangle of Galois correspondences descends to a triangle of Galois correspondences between the set of homogeneous \textbf{prime} ideals which are not equal to the irrelevant ideal, the set of \textbf{irreducible closed} sets of $ℙ^n_K$, and the set of nontrivial \textbf{irreducible closed} affine cones in $𝔸^{n+1}_K$.
  \end{enumerate}
\end{exercise}

\begin{exercise}[Homogenization and dehomogenization]
  Let $K$ be a field.
  Denote the vector space of $(n+1)$-variate polynomials of degree at most $d$ by $K[X_0,\dots,X_n]_{≤d}\coloneqq ⊕_{i ≤ d}K[X_0,\dots,X_n]_d$.
  For each $i ∈ \{0,\dots,n\}$ and $d ≥ 0$ there is a map
  \begin{align*}
    Φ_i^{(d)} : K[X_0,\dots,X_n]_d & \longrightarrow K[X_0,\dots,\widehat{X_i},\dots,X_n]_{≤d}\\
    f & \longmapsto f(X_0,\dots,1,\dots,X_n)
  \end{align*}
  called \emph{dehomogenization with respect to $X_i$}.
    Show that each $Φ^{(d)}_i$ is $K$-linear and has a $K$-linear inverse $Ψ_i^{(d)}$ called \emph{homogenization with respect to $X_i$}.
      By identifying $K[X_0,\dots,X_n]$ with a subset of $K[X_0,\dots,X_n][X_i^{-1}]$ this inverse can be written as $Ψ^{(d)}_i(f)=X_i^d ⋅ f(\tfrac{X_0}{X_i},\dots,\tfrac{X_n}{X_i})$.
\end{exercise}

%\begin{exercise}[The variety of nilpotent matrices]
%  Let $K$ be a field and identify the space $\Mat_n(K)$ of $n × n$ matrices with affine space $𝔸^{n^2}_K$.
%  Denote its coordinate ring by $R_n \coloneqq K[X_{i,j}]_{i,j ∈ \{1,\dots,n\}}$.
%  %and write $\Nil_n(K)$ for the set of nilpotent $n × n$ matrices.
%  Prove:
%  \begin{enumerate}
%    \item The characteristic polynomial of a matrix $A ∈ \Mat_n(R)$ over any commutative ring $R$ can be written as
%      $$
%      χ_A(t) \coloneqq \det(t⋅𝟙 - A) = \sum_{i = 0}^n (-1)^{n-i}s_{n-i}(A)t^i ∈ R[t]
%      $$
%      where $s_{n-i}(A) \coloneqq \tr(⋀^iA)$ is the sum of all principal $i$-minors of $A$.
%    \item For $G$ the \emph{generic $n × n$ matrix} over $K$
%      $$
%      G \coloneqq
%      \begin{bmatrix}
%        X_{1,1} & \cdots & X_{1,n}\\[0.3em]
%        \vdots & \ddots & \vdots\\[0.3em]
%        X_{n,1} & \cdots & X_{n,n}
%      \end{bmatrix}
%      ∈ \Mat_n(R_n),
%      $$
%      the ideal $(s_1(G),\dots,s_n(G))⊆R_n$ is a prime ideal.
%    \item The set $\Nil_n(K)⊆\Mat_n(K)$ of nilpotent $n × n$ matrices is an irreducible variety.
%    \item The dimension of $\Nil_n(K)$ is $n^2 - n$.
%    \item The degree of $\Nil_n(K)$ is $n!$.
%  \end{enumerate}
%\end{exercise}

\end{document}
