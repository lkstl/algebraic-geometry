\documentclass{exercises}

\DeclareMathOperator{\ev}{ev}
\DeclareMathOperator{\im}{im}
\DeclareMathOperator{\Idl}{Idl}
\DeclareMathOperator{\Rad}{Rad}
\DeclareMathOperator{\Fix}{Fix}
\DeclareMathOperator{\Spec}{Spec}
\DeclareMathOperator{\Specm}{Spec_m}

\addbibresource{exercises.bib}

\begin{document}
\maketitle{Universität Augsburg}{Lehrstuhl für Algebra und Zahlentheorie}{Wintersemester 2021/22}{Dr.~Marco Ramponi}{Lukas Stoll, M.Sc.}{Algebraic Geometry I}

\begin{exercise}[Galois connections]
  A \emph{partially ordered set}, or short \emph{poset}, is a set $A$ equipped with a reflexive, transitive and antisymmetric relation $≤$.
  A map $f:A → B$ between partially ordered sets is \emph{order preserving} if $a ≤_A a'$ implies $f(a) ≤_B f(a')$.

  A \emph{Galois connection} between two partially ordered sets is a pair of order preserving maps $g:(B,≤_B) ⇆ (A,≤_A):f$ which satisfies the following \emph{adjunction property} for all $a ∈ A$, $b ∈ B$:
  $$
  f(a) ≤_B b \quad\Longleftrightarrow\quad a ≤_A g(b)
  $$
  \begin{enumerate}
    \item Let $X$, $Y$ be sets and $⊥ ⊆ X × Y$ a relation from $X$ to $Y$.
      Prove that the following maps form a Galois connection between the power sets of $X$ and $Y$.
      \begin{align*}
        𝒱_R : (𝒫(X),⊇) &\leftrightarrows (𝒫(Y),⊆) : ℐ_R\\
        S & \longmapsto \{y ∈ Y \mid ∀s ∈ S:s \perp y\}\\
        \{x ∈ X \mid ∀t ∈ T:x \perp t\} & \longmapsfrom T
      \end{align*}
      {\scriptsize Note that $≤_A$ is the subset relation $⊆$ and $≤_B$ the reverse subset relation $⊇$, making order preserving maps between the two powersets \emph{inclusion reversing}.}
    \item Let $K$ be a field.
      Show the existence of a Galois connection
      $$
      𝒱:(𝒫(K[x_1,\dots,x_n]),⊇) ⇆ (𝒫(𝔸^n_K),⊆):ℐ
      $$
      between the power sets of the polynomial ring $K[x_1,\dots,x_n]$ and the affine space $𝔸^n_K$, such that $𝒱(S)$ is the vanishing set of $S⊆K[x_1,\dots,x_n]$ and $ℐ(T)$ the vanishing ideal of $T⊆𝔸^n_K$.
    \item Let $\Idl(R)$ denote the set of ideals of a commutative ring $R$.
      Show that $ℐ(T)$ is an ideal for every $T⊆𝔸^n_K$ and conclude that the above Galois connection descends to a Galois connection $𝒱:(\Idl(K[x_1,\dots,x_n]),⊇) ⇆ (𝒫(𝔸^n_K),⊆):ℐ$.
    \item Let $\Rad(R)$ denote the set of radical ideals of a commutative ring $R$.
      Show that $ℐ(T)$ is radical for every $T⊆𝔸^n_K$ and conclude that the above Galois connection descends to a Galois connection $𝒱:(\Rad(K[x_1,\dots,x_n]),⊇) ⇆ (𝒫(𝔸^n_K),⊆):ℐ$.\\
      {\scriptsize An ideal $𝖆 ⊆ R$ is \emph{radical} if $∀f ∈ R : (∃n ∈ ℕ : f^n ∈ 𝖆) ⇒ f ∈ 𝖆$.}
  \end{enumerate}
\end{exercise}

\begin{exercise}[Closure operators]
  A \emph{closure operator} on a poset $A$ is an order preserving map $◯:A → A$ which is \emph{idempotent}, $◯(◯(a)) = ◯(a)$, and \emph{extensive}, $a ≤ ◯(a)$.

  Let $g:B ⇆ A:f$ be a Galois connection.
  \begin{enumerate}
    \item Show that $g∘f$ is a closure operator on $(A,≤_A)$.
    \item Show that $f∘g$ is a closure operator on $(B,≥_B)$.
      {\scriptsize Mind the reverse ordering!}
  \end{enumerate}
  Let $R$ be a commutative ring.
  Prove that the following maps are closure operators.
  \begin{enumerate}[start=3]
    \item The map $(\_):𝒫(R) → 𝒫(R)$ sending a subset $S⊆R$ to the ideal it generates. 
      $$
      (S)\coloneqq \{r_1s_1 + \dots + r_ns_n \mid n ∈ ℕ,\, r_i ∈ R,\, s_i ∈ S\}
      $$
    \item The map $\sqrt{-}:\Idl(R) → \Idl(R)$ sending an ideal $𝖆⊆R$ to its radical.
      $$
      \sqrt{𝖆}\coloneqq\{r ∈ R \mid ∃n ∈ ℕ : r^n ∈ 𝖆\}
      $$
  \end{enumerate}
  Let $K$ be a field.
  Prove:
  \begin{enumerate}[start=5]
    \item $𝒱((S))=𝒱(S)$ for every subset $S⊆K[x_1,\dots,x_n]$.
    \item $𝒱(\sqrt{I})=𝒱(I)$ for every ideal $I⊆K[x_1,\dots,x_n]$.
  \end{enumerate}
\end{exercise}

\begin{exercise}[Galois correspondences]
  A Galois connection $g:B ⇆ A:f$ is a \emph{Galois correspondence} if $f$ and $g$ are mutual inverses.
  \begin{enumerate}
    \item Show that a Galois connection induces a Galois correspondence between the the sets of fixed points of its associated closure operators.
      \begin{equation*}
        \begin{tikzcd}
          (\Fix(f∘g),≤_B)
          \arrow[r,yshift=-2.5,"g"']
          & (\Fix(g∘f),≤_A)
          \arrow[l,yshift=2.5,"f"', "\sim" yshift=0.5pt]
        \end{tikzcd}
      \end{equation*}
      {\scriptsize A fixed point of an endomap $s:X → X$ is an element $x ∈ X$ with $s(x)=x$.}
  \end{enumerate}
  Let $K$ be an algebraically closed field.
  \begin{enumerate}[start=2]
    \item Find a Galois correspondence between the radical ideals of $K[x_1,\dots,x_n]$ and the affine varieties in $𝔸^n_K$.
      {\scriptsize Use Hilbert's Nullstellensatz.}
    \item Show that this Galois correspondence restricts to a Galois correspondence between the maximal ideals of $K[x_1,\dots,x_n]$ and the points in $𝔸^n_K$.
  \end{enumerate}
\end{exercise}

\begin{exercise}[Joins and meets]
  Let $(A,≤)$ be a poset and $M⊆A$.
  An upper bound $a ∈ A$ of $M$ (that is, $m ≤ a$ for all $m ∈ M$) is called a \emph{join} or \emph{supremum} of $M$ if it satisfies the following universal property:
  $$
  ∀a' ∈ A: (∀m ∈ M:m ≤ a') ⇒ a ≤ a'
  $$
  Dually $a$ is a \emph{meet} or \emph{infimum} of $M$ if it is a join of $M$ in the poset $(A,≥)$ with reverse ordering.
  If each finite subset has a join and a meet, $A$ is called a \emph{lattice}.
  If this holds for all subsets, the lattice is said to be \emph{complete}.

  \begin{enumerate}
    \item Show that meets and joins are unique if they exist.  In case of existence, write $⋁ M$ for the join of $M$ and $⋀ M$ for its meet.
    \item Show that the following are equivalent for a lattice $A$:
      \begin{enumerate}
        \item $A$ is complete.
        \item Every subset of $A$ has a meet.
        \item Every subset of $A$ has a join.
      \end{enumerate}
    \item Let $◯:A → A$ be a closure operator on a poset $(A,≤)$.
    Let $M⊆\Fix(◯)$ and regard $\Fix(◯)$ as a subposet of $(A,≤)$.
      Prove:
      \begin{enumerate}
        \item A meet of $M$ in $A$ is a fixed point of $◯$.
          Moreover, it is the meet of $M$ in $\Fix(◯)$.
        \item If $M$ has a join in $A$, its image under $◯$ is the join of $M$ in $\Fix(◯)$.
        \item Deduce that $\Fix(◯)$ is a complete lattice if $A$ is.
        \item Prove that $◯(a) = ⋀ \{m ∈ \Fix(◯) \mid a ≤ m\}$ for all $a ∈ A$.
      \end{enumerate}
    \item Compute meet and join of
      \begin{enumerate}
        \item a set $M⊆𝒫(X)$ of subsets of a set $X$,
        \item a set $M⊆\Idl(R)$ of ideals of a commutative ring $R$,
        \item a set $M⊆\Rad(R)$ of radical ideals of a commutative ring $R$,
      \end{enumerate}
      and conclude that $𝒫(X)$, $\Idl(X)$ and $\Rad(X)$ are complete lattices.
  \end{enumerate}
\end{exercise}

\begin{exercise}[Preservation of joins and meets]
  Let $g:B ⇆ A:f$ be a Galois connection.
  Show:
  \begin{enumerate}
    \item The right adjoint $g$ preserves meets: $g \left( ⋀ M \right) = ⋀ g(M)$ for all $M⊆B$.
    \item The left adjoint $f$ preserves joins: $f \left( ⋁ N \right) = ⋁ f(N)$ for all $N⊆A$.
  \end{enumerate}
  Let $K$ be a field, $(𝖆_i)_{i ∈ I}$ a family of ideals of $K[x_1,\dots,x_n]$, and $(T_i)_{i ∈ I}$ a family of subsets of $𝔸^n_K$.
  Prove the following equalities:
  \begin{enumerate}[start=3]
    \item $𝒱(∑𝖆_i) = 𝒱(⋃𝖆_i) = ⋂𝒱(𝖆_i)$.
    \item $ℐ(⋃ T_i) = ⋂ℐ(T_i)$.
  \end{enumerate}
  Finally, suppose $𝖆,𝖇 ⊆ K[x_1,\dots,x_n]$ are ideals.
  \begin{enumerate}[start=5]
    \item Prove that $𝒱(𝖆 ∩ 𝖇) = 𝒱(𝖆 ⋅ 𝖇) = 𝒱(𝖆) ∪ 𝒱(𝖇)$.
  \end{enumerate}
\end{exercise}

\begin{exercise}[Ideal operations]
  Let $R$ be a commutative ring and $𝖆,𝖇,𝖈⊆R$ ideals.
  Recall the definition of the \emph{ideal sum}
  $$
  𝖆 + 𝖇 \coloneqq (𝖆 ∪ 𝖇),
  $$
  the \emph{ideal product}
  $$
  𝖆 ⋅ 𝖇 \coloneqq (\{a ⋅ b ∈ R\mid a ∈ 𝖆, b ∈ 𝖇\}),
  $$
  and the \emph{ideal quotient}
  $$
  𝖆 : 𝖇 \coloneqq \{r ∈ R \mid r ⋅ 𝖇 ⊆ 𝖆\}.
  $$
  Show:
  \begin{enumerate}
    \item The intersection $𝖆 ∩ 𝖇$ is an ideal but the union $𝖆 ∪ 𝖇$ is not in general.
    \item The quotient $𝖆 : 𝖇$ is an ideal but $\{a ⋅ b ∈ R \mid a ∈ 𝖆, b ∈ 𝖇\}$ is not in general.
    \item There is a chain of inclusions $𝖆 ⋅ 𝖇 ⊆ 𝖆 ∩ 𝖇 ⊆ 𝖆 ⊆ 𝖆 + 𝖇$.
      Find counterexamples to show that the inclusions are strict in general.
    \item $\sqrt{𝖆 ⋅ 𝖇} = \sqrt{𝖆 ∩ 𝖇} = \sqrt{𝖆} ∩ \sqrt{𝖇}$.
    \item $\sqrt{𝖆 + 𝖇} = \sqrt{\sqrt{𝖆} + \sqrt{𝖇}}$.
  \end{enumerate}
\end{exercise}

\begin{exercise}[Chinese remainder theorem, from {\cite[Ex.~(1.14)]{AltmanKleiman2013}}]
  Let $R$ be a ring and $𝖆,𝖇⊆R$ ideals.
  Two ideals $𝖆,𝖇⊆R$ are \emph{comaximal} if $𝖆 + 𝖇 = R$.
  \begin{enumerate}
    \item Let $𝖆,𝖇$ be comaximal ideals.
      Prove:
      \begin{enumerate}
        \item $𝖆⋅𝖇 = 𝖆 ∩ 𝖇$.
        \item $R/(𝖆⋅𝖇) ≅ (R/𝖆) × (R/𝖇)$.
      \end{enumerate}
    \item Let $𝖆$ be comaximal to $𝖇$ and $𝖇'$.
      Prove that $𝖆$ is comaximal to $𝖇⋅𝖇'$.
    \item Let $𝖆$, $𝖇$ be comaximal and $m,n≥1$.
      Prove that $𝖆^m$ and $𝖇^n$ are comaximal.
    \item Let $𝖆_0,\dots,𝖆_n$ be pairwise comaximal.
      Prove:
      \begin{enumerate}
        \item $𝖆_0$ and $𝖆_1 \cdots 𝖆_n$ are comaximal.
        \item $𝖆_0∩\dots∩𝖆_n = 𝖆_0 \cdots 𝖆_n$.
        \item $R/(𝖆_0\cdots 𝖆_n) ≅ (R/𝖆_0) × \cdots × (R/𝖆_n)$.
      \end{enumerate}
  \end{enumerate}
\end{exercise}

\begin{exercise}[Counting monomials]
  Let $K$ be a field.
  Prove that the number of monomials of degree $d$ in $n$ variables over $K$ is given by the binomial coefficient $\binom{d+n-1}{n-1}$.\\
  {\scriptsize This is the dimension of the vector space $K[x_1,\dots,x_n]_d$ of polynomials over $K$ in $n$ variables of degree $d$.}
\end{exercise}

\begin{exercise}[Fun with monomial orders]
  List all $20$ monomials in three variables $K[x,y,z]$ of degree $≤3$ 
  \begin{enumerate}
    \item in the lexicographical order,
    \item in the deg-lex order,
    \item in the deg-rev-lex order.
  \end{enumerate}
\end{exercise}

\begin{exercise}[The Zariski topology]
  Let $(A,≤)$ be a partially ordered set.
  A closure operator $◯:A → A$ is called \emph{topological} or \emph{Kuratowski} if it preserves finite joins.
  \begin{enumerate}
    \item Suppose $A$ is the power set of a set $X$.
      Show that any topological closure operator $◯$ on $A$ induces a topology on $X$ whose closed sets are the fixed points of $◯$.
      Conversely, show that any topology on $X$ induces a topological closure operator on its power set.
    \item Let $K$ be a field.
      Show that the closure operator $𝒱∘ℐ$ on the power set of $𝔸^n_K$ is topological and that its closed sets are the affine varieties.
    \item More generally, let $R$ be a commutative ring.
      Find a Galois connection between the power set of $R$ and the power set of its spectrum whose associated closure operator is topological.
      Show that it induces the Zariski topology.
    \item In the case that $R=K[x_1,\dots,x_n]$ for a field $K$, find a map $𝔸^n_K → \Spec R$ which is a subspace inclusion with respect to the Zariski topologies.
  \end{enumerate}
\end{exercise}

\begin{exercise}[Ring maps induce continuous functions]
  Let $f:R → S$ be a ring homomorphism.
  Given a prime ideal $𝖕⊆S$, let $f^*𝖕 \coloneqq f^{-1}𝖕$.
  \begin{enumerate}
    \item Show that this defines a continuous map $f^*:\Spec S → \Spec R$.
    \item Prove that $(g∘f)^*=f^*∘g^*$ for two composable ring maps $f$ and $g$.
  \end{enumerate}
\end{exercise}

\begin{exercise}[Polynomials and polynomial functions]
  Let $W⊆𝔸^n_K$ be an affine variety over a field $K$ and denote the $K$-algebra of maps $W → K$ by $K^W$.
  There is  a unique morphism of $K$-algebras $\ev_W:K[X_1,\dots,X_n] → K^W$ sending $X_i$ to the restriction of the $i$-th projection $𝔸^n_K → K$ to $W$.
  An element in the image of this morphism is called a \emph{polynomial function}.
  \begin{enumerate}
    \item Confirm that the kernel of $\ev_W$ is given by the vanishing ideal $ℐ(W)$ of $W$.
    \item Deduce that $\ev_W$ is injective only if $W=𝔸^n_K$.
    \item Show that $\ev_{𝔸^n_K}$ is injective only if $K$ is infinite.
    \item Does the converse hold?
  \end{enumerate}
\end{exercise}

\begin{exercise}[Connected and irreducible spaces]
  Let $X$ be a topological space.
  \begin{enumerate}
    \item $X$ is \emph{connected} if for any finite disjoint decomposition $⨆_{i=1}^nZ_i$ into closed subsets $Z_i⊆X$ there is an $i$ with $X=Z_i$.
      A subspace $Y⊆X$ which is maximal with respect to inclusion among all connected subspaces of $X$ is called a \emph{connected component} of $X$.
      Prove:
      \begin{enumerate}
        \item A subspace of $X$ is connected if and only if its closure is.
        \item Every connected subspace of $X$ is contained in a unique connected component.
        \item $X$ is the disjoint union of its connected components.
      \end{enumerate}
    \item $X$ is \emph{irreducible} if for any finite decomposition $⋃_{i=1}^nZ_i$ into closed subsets $Z_i⊆X$ there is an $i$ with $X=Z_i$.
      A subspace $Y⊆X$ which is maximal with respect to inclusion among all irreducible subspaces of $X$ is called an \emph{irreducible component} of $X$.
      \begin{enumerate}
        \item A subspace of $X$ is irreducible if and only if its closure is.
        \item Every irreducible subspace of $X$ is contained in a unique irreducible component.
        \item $X$ is the union of its irreducible components.
      \end{enumerate}
    \item Show that an irreducible space is connected.
    \item Show that the empty space is neither connected nor irreducible.
    \item Show that $ℝ^n$ is connected but not irreducible in the Euclidean topology.
    \item Show that the one-point space $∗$ is irreducible.
  \end{enumerate}
\end{exercise}

\printbibliography
\end{document}
