\documentclass{exercises}

\DeclareMathOperator{\im}{im}
\DeclareMathOperator{\Idl}{Idl}
\DeclareMathOperator{\Rad}{Rad}
\DeclareMathOperator{\Fix}{Fix}

\begin{document}
\maketitle{Universität Augsburg}{Lehrstuhl für Algebra und Zahlentheorie}{Wintersemester 2021/22}{Dr.~Marco Ramponi}{Lukas Stoll, M.Sc.}{Algebraic Geometry I}

\begin{exercise}[Galois connections]
  A \emph{partially ordered set}, or short \emph{poset}, is a set $A$ equipped with a reflexive, transitive and antisymmetric relation $≤$.
  A map $f:A → B$ between partially ordered sets is \emph{order preserving} if $a ≤_A a'$ implies $f(a) ≤_B f(a')$.

  A \emph{Galois connection} between two partially ordered sets is a pair of order preserving maps $g:(B,≤_B) ⇆ (A,≤_A):f$ which satisfies the following \emph{adjunction property} for all $a ∈ A$, $b ∈ B$:
  $$
  f(a) ≤_B b \quad\Longleftrightarrow\quad a ≤_A g(b)
  $$
  \begin{enumerate}
    \item Let $X$, $Y$ be sets and $⊥ ⊆ X × Y$ a relation from $X$ to $Y$.
      Prove that the following maps form a Galois connection between the power sets of $X$ and $Y$.
      \begin{align*}
        𝒱_R : (𝒫(X),⊇) &\leftrightarrows (𝒫(Y),⊆) : ℐ_R\\
        S & \longmapsto \{y ∈ Y \mid ∀s ∈ S:s \perp y\}\\
        \{x ∈ X \mid ∀t ∈ T:x \perp t\} & \longmapsfrom T
      \end{align*}
      {\scriptsize Note that $≤_A$ is the subset relation $⊆$ and $≤_B$ the reverse subset relation $⊇$, making order preserving maps between the two powersets \emph{inclusion reversing}.}
    \item Let $K$ be a field.
      Show the existence of a Galois connection
      $$
      𝒱:(𝒫(K[x_1,\dots,x_n]),⊇) ⇆ (𝒫(𝔸^n_K),⊆):ℐ
      $$
      between the power sets of the polynomial ring $K[x_1,\dots,x_n]$ and the affine space $𝔸^n_K$, such that $𝒱(S)$ is the vanishing set of $S⊆K[x_1,\dots,x_n]$ and $ℐ(T)$ the vanishing ideal of $T⊆𝔸^n_K$.
    \item Let $\Idl(R)$ denote the set of ideals of a commutative ring $R$.
      Show that $ℐ(T)$ is an ideal for every $T⊆𝔸^n_K$ and conclude that the above Galois connection descends to a Galois connection $𝒱:(\Idl(K[x_1,\dots,x_n]),⊇) ⇆ (𝒫(𝔸^n_K),⊆):ℐ$.
    \item Let $\Rad(R)$ denote the set of radical ideals of a commutative ring $R$.
      Show that $ℐ(T)$ is radical for every $T⊆𝔸^n_K$ and conclude that the above Galois connection descends to a Galois connection $𝒱:(\Rad(K[x_1,\dots,x_n]),⊇) ⇆ (𝒫(𝔸^n_K),⊆):ℐ$.\\
      {\scriptsize An ideal $𝖆 ⊆ R$ is \emph{radical} if $∀f ∈ R : (∃n ∈ ℕ : f^n ∈ 𝖆) ⇒ f ∈ 𝖆$.}
    %\item Finde eine Galoisverbindung zwischen den Zwischenkörpern $E$ einer Galoiserweiterung $L/K$ und den Untergruppen der Galoisgruppe $\mathrm{Gal}(L/K)$.
  \end{enumerate}
\end{exercise}

\begin{exercise}[Closure operators]
  A \emph{closure operator} on a poset $A$ is an order preserving map $◯:A → A$ which is \emph{idempotent}, $◯(◯(a)) = a$, and \emph{extensive}, $a ≤ ◯(a)$.

  Let $g:B ⇆ A:f$ be a Galois connection.
  \begin{enumerate}
    \item Show that $g∘f$ is a closure operator on $(A,≤_A)$.
    \item Show that $f∘g$ is a closure operator on $(B,≥_B)$.
      {\scriptsize Mind the reverse ordering!}
  \end{enumerate}
  Let $R$ be a commutative ring.
  Prove that the following maps are closure operators.
  \begin{enumerate}[start=3]
    \item The map $(\_):𝒫(R) → 𝒫(R)$ sending a subset $S⊆R$ to the ideal it generates. 
      $$
      (S)\coloneqq \{r_1s_1 + \dots + r_ns_n \mid n ∈ ℕ,\, r_i ∈ R,\, s_i ∈ S\}
      $$
    \item The map $\sqrt{-}:\Idl(R) → \Idl(R)$ sending an ideal $𝖆⊆R$ to its radical.
      $$
      \sqrt{𝖆}\coloneqq\{r ∈ R \mid ∃n ∈ ℕ : r^n ∈ 𝖆\}
      $$
  \end{enumerate}
  Let $K$ be a field.
  Prove:
  \begin{enumerate}[start=5]
    \item $𝒱((S))=𝒱(S)$ for every subset $S⊆K[x_1,\dots,x_n]$.
    \item $𝒱(\sqrt{I})=𝒱(I)$ for every ideal $I⊆K[x_1,\dots,x_n]$.
  \end{enumerate}
\end{exercise}

\begin{exercise}[Galois correspondences]
  A Galois connection $g:B ⇆ A:f$ is a \emph{Galois correspondence} if $f$ and $g$ are mutual inverses.
  \begin{enumerate}
    \item Show that a Galois connection induces a Galois correspondence between the the sets of fixed points of its associated closure operators.
      \begin{equation*}
        \begin{tikzcd}
          (\Fix(f∘g),≤_B)
          \arrow[r,yshift=-2.5,"g"']
          & (\Fix(g∘f),≤_A)
          \arrow[l,yshift=2.5,"f"', "\sim" yshift=0.5pt]
        \end{tikzcd}
      \end{equation*}
      {\scriptsize A fixed point of an endomap $s:X → X$ is an element $x ∈ X$ with $s(x)=x$.}
  \end{enumerate}
  Let $K$ be an algebraically closed field.
  \begin{enumerate}[start=2]
    \item Find a Galois correspondence between the radical ideals of $K[x_1,\dots,x_n]$ and the affine varieties in $𝔸^n_K$.
      {\scriptsize Use Hilbert's Nullstellensatz.}
    \item Show that this Galois correspondence restricts to a Galois correspondence between the maximal ideals of $K[x_1,\dots,x_n]$ and the points in $𝔸^n_K$.
  \end{enumerate}
\end{exercise}

\begin{exercise}[Joins and meets]
  Let $(A,≤)$ be a poset and $M⊆A$.
  An upper bound $a ∈ A$ of $M$ (that is, $m ≤ a$ for all $m ∈ M$) is called a \emph{join} or \emph{supremum} of $M$ if it satisfies the following universal property:
  $$
  ∀a' ∈ A: (∀m ∈ M:m ≤ a') ⇒ a ≤ a'
  $$
  Dually $a$ is a \emph{meet} or \emph{infimum} of $M$ if it is a join of $M$ in the poset $(A,≥)$ with reverse ordering.
  If each finite subset has a join and a meet, $A$ is called a \emph{lattice}.
  If this holds for all subsets, the lattice is said to be \emph{complete}.

  \begin{enumerate}
    \item Show that meets and joins are unique if they exist.  In case of existence, write $⋁ M$ for the join of $M$ and $⋀ M$ for its meet.

    \item Let $◯:A → A$ be a closure operator on a poset $(A,≤)$.
    Let $M⊆\Fix(◯)$ and regard $\Fix(◯)$ as a subposet of $(A,≤)$.
      Prove:
      \begin{enumerate}
        \item A meet of $M$ in $A$ is a fixed point of $◯$.
          Moreover, it is the meet of $M$ in $\Fix(◯)$.
        \item If $M$ has a join in $A$, its image under $◯$ is the join of $M$ in $\Fix(◯)$.
        \item Deduce that $\Fix(◯)$ is a complete lattice if $A$ is.
        \item Prove that $◯(a) = ⋀ \{m ∈ \Fix(◯) \mid a ≤ m\}$ for all $a ∈ A$.
      \end{enumerate}
    \item Compute meet and join of
      \begin{enumerate}
        \item a set $M⊆𝒫(X)$ of subsets of a set $X$,
        \item a set $M⊆\Idl(X)$ of ideals of a commutative ring $R$,
        \item a set $M⊆\Rad(X)$ of radical ideals of a commutative ring $R$.
      \end{enumerate}
%    \item Sei $g:B ⇆ A:f$ eine Galoisverbindung.
%      Zeige, dass $g$ Infima und $f$ Suprema erhält.
%      Das heißt für jede Teilmenge $M⊆B$ und $N⊆A$ ist
%      \begin{align*}
%        g \left( ⋀ M \right) = ⋀ g(M),\\
%        f \left( ⋁ M \right) = ⋁ f(M).
%      \end{align*}
%      Folgere die nachfolgenden Regeln für die Verschwindungsmenge einer Familie von Idealen $(𝖆_i)_{i ∈ I}$ des Polynomrings $K[x_1,\dots,x_n]$ über einem Körper $K$:
%      $$
%      𝒱(⋃_{i ∈ I}𝖆_i) = 𝒱(∑_{i ∈ I}𝖆_i) = ⋂_{i ∈ I}𝒱(𝖆_i)
%      $$
%    \item Sei $K$ ein Körper, $(S_i)_{i ∈ I}$ eine Familie von Teilmengen, $(𝖆_i)_{i ∈ I}$ eine Familie von Idealen und $(𝖗_i)_{i ∈ I}$ eine Familie radikaler Ideal von $K[x_1,\dots,x_n]$.
%      Zeige die folgenden Regeln für $𝒱$ und $ℐ$:
%      \begin{multicols}{2}
%        \begin{enumerate}
%          \item $𝒱(∑𝖆_i) = ⋂𝒱(𝖆_i)$
%          \item $𝒱(I∩J)=𝒱(I)∪𝒱(J)$
%        \end{enumerate}
%      \end{multicols}
  \end{enumerate}
\end{exercise}
\end{document}
